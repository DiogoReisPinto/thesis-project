%!TEX root = ../report.tex

% 
% Related work
% 


\section{A State of the Art in Frameworks for Information Technology Governance and Management}

% Example citation:
In this section we will present a set of literature references on the subjects related to this project. We will present the most important frameworks on Information Systems Management and Governance with the objective of coming up with a choice of a framework or a set of them to implement our processes for project and maintenance management.\par
Our choice is centered in three frameworks we consider the most relevant for this project: COBIT 5, ITIL V3 and PMBOK. This three frameworks provide, from different perspectives, guides and principles for IT Governance and Management, presenting processes for achieving a successful implementation of this principles in an organization.

\subsection{IT Governance and IT Management}

One important concept to define is the difference between IT Governance and IT management. They are many times confused and some authors already tried to explain the difference between the two concepts.\par
Considering the definition given by Van Grembergen \textit{et al.}, ``IT Management is focused on the internal effective supply of IT services and products and the management of present IT operations. IT Governance in turn is much broader, and concentrates on performing and transforming IT to meet present and future demands of the business (internal focus) and the business' customers (external focus).''.\par
 Considering the COBIT 5 view for this question, there is a ``clear distinction between governance and management, in the way these two disciplines encompass different types of activities, require different organizational structures and serve different purposes. Governance ensures that stakeholders' needs, conditions and options are evaluated to determine balanced, agreed-on enterprise objectives to be achieved. it sets direction through prioritisation and decision making and monitors performance and compliance against agreed-on direction and objectives.On the other hand, management plans, builds, runs and monitors activities in alignment with the direction set by the governance body to achieve the enterprise objectives.''\cite{2012cobit}\par



 Considering both definitions, we can conclude that IT Governance has a bigger dimension that IT Management, but both need to be related and complementary to achieve success inside an organization. It is not possible for an organization to have well defined and matured management processes that are no related to governance aspects, but governance needs management to achieve goals and objectives settled to achieve success.

\subsection{COBIT 5}

Control Objectives for Information and Related Technology (COBIT) is a framework created by the Information Systems Audit and Control Association (ISACA) for IT Management and IT Governance.\par
As stated by ISACA,``COBIT 5 provides a comprehensive framework that assists enterprises in achieving their objectives for the governance and management of enterprise IT. Simply stated, it helps enterprises create optimal value from IT by maintaining a balance between realizing benefits and optimizing risk levels and resource use'' \cite{2012cobit}. The framework is built on five basic principles:

\begin{itemize}
  \item Meeting the Stakeholders Needs 
  \item Covering the Enterprise End-to-end
  \item Applying a Single, Integrated Framework
  \item Enabling a Holistic Approach
  \item Separating Governance from Management
\end{itemize}


It also defines seven enablers, explained by COBIT as factors that, individually and collectively, influence whether governance and management over enterprise will work or not. This enablers can be categorized as:

\begin{itemize}
  \item Principles, Policies and frameworks 
  \item Processes 
  \item Organizational structures
  \item Culture, ethics and behavior 
  \item Information
  \item Services, infrastructure and applications
  \item People, skills and competencies
\end{itemize}

\begin{figure}
\centering
\includegraphics[width=0.7\textwidth]{img/Enablers.png}
\caption{COBIT 5 enablers}
\end{figure}

Figure 2 presents the COBIT 5 enablers previous defined and how they relate among themselves int terms of its importance for organization. Each enabler has stakeholders, a set of goals, a life cycle and can be defined good practices for each one.\par

\begin{figure}
\centering
\includegraphics[width=0.9\textwidth]{img/COBITProcesses.jpg}
\caption{COBIT 5 domains}
\end{figure}

Considering figure 3, COBIT 5 process reference model considers two big domains of processes: Governance and Management. The governance domain contains five processes in the domain Evaluate, Direct and Monitor(EDM). The management domain has four internal domains of processes: Align, Plan and Organise(APO), Build, Acquire and Implement(BAI), Deliver, Service and Support (DSS) and Monitor, Evaluate and Assess(MEA).\par
All processes for management and governance are presented in the appendix and all the implementation details explained in COBIT 5: Enabling Processes, A detailed reference guide to the processes defined in the COBIT 5 process reference model. This includes the COBIT 5 goals cascade, a process model explanation, governance and management practices, and the process reference model\cite{2012cobitEP}.\par
Relating to other frameworks and standards, COBIT tries to establish a framework that is compliant with the most widely accepted standards in IT governance and management. In figure 4 we can see the standards COBIT 5 relates by processes domain, with special attention to ITIL V3\cite{itilSS,itilSD,itilSO,itilST,itilCSI}, ISO/IEC 20000\cite{ISO20000-1}, PMBOK\cite{pmbok5} and CMMI\cite{cmmi}, that are closely related to this project's problem. This compliance with other standards is fundamental for a widely adoption of COBIT 5, in the way it tries to establish goals, metrics, practices, roles, inputs and outputs for each process, making it necessary being compliant with international standards. This will improve COBIT application and acceptance on organizations.\par


\subsubsection{CMMI}

CMMI (Capability Maturity Model Integration), first developed at the Software Engineering Institute at Carnegie Mellon University and currently operated by the CMMI Institute, consists on a set of practices and process improvement goals that organizations can use to evaluate and improve its processes. The CMMI framework provides all structures to produce CMMI models, training and appraisal methods.
CMMI, currently on version 1.3, defines areas of interest that group collections of CMMI components for models, training and appraisal construction. The three areas of interest for CMMI are:

\begin{itemize}

\item CMMI for Acquisition (CMMI-ACQ) - Provides guidance to organizations that manage the supply chain to acquire and integrate products and services to meet the needs of the customer.

\item CMMI for Development (CMMI-DEV) - Provides guidance to improve the effectiveness, efficiency, and quality of their product development work. Used for process improvement in organizations that develop products.

\item CMMI for Services (CMMI-SVC) - Provides guidance to organizations that establish, manage, and deliver services that meet the needs of customers and end users.

\end{itemize}

For this areas of interest, CMMI defines 22 core process areas, covering processes that are fundamental to improve the organization processes. Each one of the areas is a set of related practices that, when implemented, will achieve goals important for process improvement. The process areas are presented in Appendix section XXX.\par
CMMI supports two types of improvement paths, depending on the organization objectives. One is used for the organization to improve processes related to a specific process area, named the continuous representation, and the other one is used for organizations to improve a set of relating processes by incrementally consider sets of process areas, named the staged representation. As stated by CMMI, the continuous representation will allow to achieve capability levels and the staged representation will achieve maturity levels.\par
For the staged representation, the following capability levels are defined:

\begin{itemize}

\item Level 0 - Incomplete - Process not performed or only partially performed. Specific goals of process area not achieved.

\item Level 1 - Performed - Process performs the needed work to produce work products and satisfies the specific goals of the process area.

\item Level 2 - Managed - Performed process that is planned and executed according with policy.

\item Level 3 - Defined - Managed process that is tailored from the organization's set of standard processes according to the organization's tailoring guidelines.
 
\end{itemize}

For the continuous representation, the maturity levels defined are:

\begin{itemize}

\item Level 1 - Initial - Chaotic and Ad-Hoc processes. Organization does not provide a stable environment to support processes. Organizations are characterized by easily abandoning its processes and being unable to repeat them.

\item Level 2 - Managed - Projects establish the foundation for an organization to become an effective acquirer of needed capabilities by institutionalizing select project management and acquisition engineering processes. Processes, projects, products and services are managed and periodically evaluated. Processes are maintained in times of crisis.

\item Level 3 - Defined - Acquirers use defined processes for managing projects and suppliers. Project management and acquisition practices are embedded in the standard process set. Organization set of standards is established and improved over time. Processes described more rigorously than in level 2.

\item Level 4 - Quantitatively Managed - Acquirers establish quantitative objectives for quality and process performance and use them as criteria in managing processes. Process performance becomes predictable, controlled using statistical and quantitative techniques for prediction.

\item Level 5 - Optimizing - Process performance continuous improvement based on the organizations' objectives and performance needs comprehension. optimization is achieved through incremental and innovative process and technological improvements.

\end{itemize}

CMMI does not provide certification to organizations. Instead, it can be used to conduct appraisals, measuring the organization progress and earning maturity or capability levels face to CMMI Levels. These appraisals can be used for comparing the current practices with the CMMI best practices, finding areas to improve, for outside accreditation by suppliers and other stakeholders of conformance to a specific level of CMMI and to meets contractual requirements.\par
To conduct an CMMI appraisal, the organization must follow the appraisal Requirements for CMMI document [INSERT REFERENCE HERE] and must use an CMMI model and an appraisal method conformant with the appraisal Requirements for CMMI, as the SCAMPI appraisal method [INSERT REFERENCE HERE]\par


\subsubsection{CMMI importance for COBIT 5}

CMMI can be used by COBIT for process improvement purposes. Considering COBIT as a framework for governance and management processes, providing guidance and control, it is fundamental for it to also consider improvement as an important part of these processes. Also, and considering COBIT application for this project, we will apply its guidance on a real-case organization, that deals with suppliers and other stakeholders, many time interested in the processes level of maturity conducted by the organization, in accordance to CMMI levels. Considering this, its fundamental for COBIT application to consider the CMMI body of knowledge and application.\par
COBIT 5 presents an process capability model, based on ISO/IEC 15504 Software Engeneering - Process Assessment standard, presented in section XXX. This standard has many similarities and guidelines presented by CMMI, but is not as widely accepted as this one. In fact, it has appeared after CMM, that was been replaced by the CMMI, and cannot face with some benefits CMMI presents face to this standard.\par
COBIT 5 also considers CMMI as a related framework, as stated in figure XXX, and identifies, as stated in [REFERENCE HERE], that some areas and domains are covered by CMMI, namely application-building and acquisition-related processes in the Build And Acquire domain and some organizational and quality-related processes from the Acquire, Plan and Organize domain.\par


\subsubsection{COBIT 5 process capability model}

COBIT 5 includes a process capability model based on ISO/IEC 15504 Software Engineering - Process Assessment standard.\cite{ISO15504} This model allow to measure the current level of maturity of enterprise processes, presenting the gap between the current level and the desired one the enterprise wants to achieve. This new capability model is an improvement of the previous on COBIT 4.1 \cite{cobit4}, being more simplified and compliant with a generally accepted process assessment standard.\par


\subsubsection{COBIT 5 critical analysis}

The COBIT 5 is one of the most interesting frameworks widely accepted by organizations in the IT management and Governance area. It arises as the main framework for establishing processes to guide us on management and governance and establish ways to control them. However, it is a complex framework that needs time and practice to be fully implemented.\par
For this project, we will consider only the domains relevant for our objectives, making a selection of the processes we pretend to implement. This will allow us to get the bigger value COBIT has to offer, making it possible, in the time-frame available, achieve our implementation objectives.\par
One important aspect of the use of COBIT is that it provides a more business and strategic view of IT on organizations, presenting a lack of operational approach to some themes that are relevant for our project. To overcome this, we will analyze a more operational framework on IT service management, the ITIL V3 framework\cite{itilIntro,itilSS,itilST,itilSD,itilSO,itilCSI} and a project management guide considered by the main specialists on the area as the reference for project management, the Project Management Book of Knowledge\cite{pmbok5}.\par

\begin{figure}
\centering
\includegraphics[width=0.9\textwidth]{img/COBITOtherFrameworks.png}
\caption{COBIT 5 coverage on other frameworks}
\end{figure}

\subsection{ITIL V3}

First developed in the 1980s by the actual Office of Government Commerce (OGC), a branch of the British Government, ITIL defines processes for IT Service management at a high level. Each organization that intents to apply ITIL for service management should adapt and implement it in the most suitable manner to accomplish  particular objectives and needs.\cite{hill2006combine}\par
On the last years, ITIL became an important standard worldwide for organizations as the guideline for IT service management (ITSM) processes. Its guidance can be used to transform service management capabilities into strategic assets, that will become fundamental to build distinctiveness to the concurrents and deliver services with higher performance and costumer satisfaction.\par
The ITIL service management practices are comprised of three main sets of products and services: ITIL service management practices (core guidance), ITIL service management practices (guidance specific to industry sectors, organization types, operating models and technology architectures) and ITIL web support services.\par

\begin{figure}
\centering
\includegraphics[width=0.7\textwidth]{img/ITILVolumes.png}
\caption{ITIL V3 volumes}
\end{figure}

The core set, presented in figure 5 and the one we will consider for this project, consists of six publications: Introduction to ITIL Service Management Practices, Service Strategy, Service Design, Service Transition, Service Operation and Continual Service Improvement. Each one of this volumes share the same conceptual structure, being composed by practice fundamentals and principles, life cycle processes and activities, supporting organization structures and roles, technology considerations, practice implementation and challenges, risks and critical success factors.

\subsubsection{Service Strategy} 

Service Strategy volume provides guidance on achieving strategic assets by improving actual service management capabilities. It presents principles for service management that are important for developing and implementing service management policies, guidelines and processes across the ITIL Service Life cycle.\cite{itilSS} The processes included in Service Strategy volume are:

\begin{itemize}
  \item Financial Management
  \item Service Portfolio Management 
  \item Demand Management
\end{itemize}

\subsubsection{Service Design}

Service Design volume provides guidance for the design and development of services and service management practices. As stated on this volume, ``It covers design principles and methods for converting strategic objectives into portfolios of services and service assets. The scope of Service Design is not limited to new services. It includes the changes and improvements necessary to increase or maintain value to customers over the life cycle of services, the continuity of services, achievement of service levels, and conformance to standards and regulations.''\cite{itilSD} The processes included in Service Design volume are:

\begin{itemize}
  \item Service Catalogue Management
  \item Service Level Management 
  \item Capacity Management
  \item Availability Management
  \item IT service Continuity Management
  \item Information Security Management 
  \item Supplier Management
  \item Application Management
  \item Data and Information Management
  \item Business Service Management
\end{itemize} 

\subsubsection{Service Transition} 

Service Transition volume provides guidance for implementing and improve processes for transitioning services in developing or maintaining operations into live service operation. As stated in this volume, ``This publication provides guidance on how the requirements of Service Strategy encoded in Service Design are effectively realized in Service Operation while controlling the risks of failure and disruption.''\cite{itilST} The processes included in Service Transition volume are:

\begin{itemize}
  \item Change Management
  \item Service asset and Configuration Management
  \item Release and deployment Management
  \item Knowledge Management
  \item Stakeholder Management
  \item Transition Planning 
  \item Support and Service Evaluation 
\end{itemize} 

\subsubsection{Service Operation} 

Service Operation volume presents practices and operations needed to deal with the day-to-day operation of services that are already in live service operation. Pretends to guide on the effectiveness and efficiency achievement on delivering and supporting services, ensuring that it creates value for the customer and for the organization. It is a fundamental capability because is directly connected with IT management and organization's strategic objectives . As stated in the volume, ``Guidance is provided on how to maintain stability in service operations, allowing for changes in design, scale, scope and service levels.''\cite{itilSO} The processes included in Service Operation volume are:

\begin{itemize}
  \item Event Management
  \item Incident Management
  \item Request Management
  \item Problem Management
  \item Access management
\end{itemize} 

\subsubsection{Continual Service Improvement}

As explained in the Continual Service Improvement volume, it ``provides instrumental guidance in creating and maintaining value for customers through better design, transition and operation of services. It combines principles, practices and methods from quality management, change management and capability improvement. Organizations learn to realize incremental and large-scale improvements in service quality, operational efficiency and business continuity.''\cite{itilST} This volume also pretends to link service improvement with the guidelines expressed in all other volumes, making it cover the all service life cycle. The processes included in Continual Service Improvement volume are:

\begin{itemize}
  \item The 7-Step Improving Process
  \item Service Level Management
\end{itemize} 

\par Not different from COBIT,  ITIL takes public frameworks and standards as a form of the organization to have advantage on the market. Organizations should build their proprietary knowledge considering  all the knowledge provided by public frameworks and standards. In addition with this, collaboration and coordination between organizations became easier due to a set of shared practices and standards. According to a research performed by the UK Department of Trade and Industry (DTI), the value to the UK economy from standards is estimated to be about \textsterling2.5 billion per annum \cite{McNeillis01112005}.\par
For related standards and frameworks to ITIL V3, we have ISO/IEC 20000 (service management system standards), ISO/IEC 27001 (standard providing requirements for an information security management system), PMBOK (manual for a set of standard terminology and guidelines for project management)\cite{pmbok5} and COBIT\cite{2012cobit}, already presented. This are the standards we will cover for our project by being directly related to ITIL V3 and its implementation.


\subsubsection{ITIL V3 critical analysis}

One crucial aspect for the importance of ITIL on this project is the operational view that it provides for IT Service Management. ITIL tries to focus on more management details, providing a more practical guidance for implementation. It is focused on IT Service Management and presents concrete guidance for managing services during its life cycle. De Haes and Van Grembergen state that COBIT tells what to do and ITIL explains how to do it, what makes COBIT adopting a process-focused approach and ITIL a service level-oriented one \cite{ITGovAndMech}.\par
The main objective to include ITIL knowledge for this project is to provide a complementary guidance on IT management, enhancing the business oriented view of COBIT with a operational view. COBIT 5 will allow us to take advantage of this complementarity, related to the concern of ISACA to make it more compliant with other frameworks, including COBIT, on the new version (relating to COBIT 4.1). 

\subsection{PMBOK: A Guide to the Project Management Book of Knowledge}

The Guide to the Project Management Body of Knowledge provides guidance on individual projects management and the concepts inherited to it. To achieve this, it presents the processes involved in the project management and project life cycle.\cite{pmbok5}\par
This guide is considered by many professionals in all business management areas as the main reference for processes and good practices in project management, because it compiles a set of good practices that are applicable to most of the projects in most of the contexts, bringing value to all organizations and managers that use it as a reference.\par 
The processes are presented relating them with the knowledge area they belong. The PMBOK presents the following areas of project management:

\begin{itemize}
  \item Integration Management
  \item Scope Management
  \item Time Management
  \item Cost Management
  \item Quality management
  \item Human resource management
  \item Communications management
  \item Risk management
  \item Procurement management
  \item Stakeholder management
\end{itemize} 

Each group of processes is related with the life cycle of a project, being also grouped in five categories: Initiating, Planing, Executing, Monitoring and Control and Closing, directly related to the phase they are applied on the life cycle. For each process, it is presented the inputs, tools, techniques and outputs that are required to successfully implement it. All processes organized by group and area are presented in figure 6\par

\begin{figure}
\centering
\includegraphics[width=\textwidth]{img/PMBOKprocesses.png}
\caption{PMBOK processes organized by group and area of knowledge}
\end{figure}


This guide also provides some background on the project management area, defining common vocabulary and establishing concepts necessary to fully understand all processes. Presents characteristics of projects, programmes and portfolios, roles of project managers and organizational aspects that influence the management process, like organizational structures, culture, assets or stakeholders.\par 
Another important aspect of PMBOK is that is a more general framework, making it necessary to complement with other frameworks or guides when applying to a specific area. This guide only presents the general processes for project management, lacking on implementation details for specific areas, like the area of IT Management.\par 

\subsubsection{PMBOK critical analysis}

The importance of PMBOK for this project is related to its widely acceptance and adoption as the reference guide to project management by many professionals in the area, being tested and evaluated its importance in terms of good practices adopted in project management. Despite being more general and lacking in specificity to IT management, it can be complemented with the two previous frameworks presented (COBIT and ITIL), using some more detailed guidance of ITIL and COBIT to improve PMBOK focus to this project.\par
PMBOK presents processes for all phases of the life cycle of a project, being too extensive considering the theme of this project and the time-frame available. For this scope, we will only be able to consider a subset of all processes presented, being necessary, in the solution's architecture phase of this project, present a mapping between processes covered on PMBOK relating with processes of COBIT and ITIL.


\section{Relevant International Standards}

In this section we will present the international standards that are referenced or important to complement the frameworks previously presented. These standards will allow us to design processes in conformance with practices that are considerer as the reference in some areas of project management.\par
We will analyze the ISO/IEC 12207\cite{ISO12207}, an systems and software engineering international standard for Software life cycle processes. This standard will allow us to define the processes that are important to consider in the scope of this project. It will provide an high-level process reference for the complete life cycle of a software system\par
Relating to COBIT, we have standards that are directly related to it, as the ISO/IEC 20000\cite{ISO20000-1,ISO20000-2,ISO20000-3,ISO20000-4,ISO20000-5}, a set of international standards for IT service management, and ISO/IEC 27000\cite{ISO27000}, a set of international standards for information security management systems. COBIT also relates with CMMI\cite{cmmi}, the Capability Maturity Model Integration that is a framework for process improvement in business comprising models, appraisal methods, and training.\par
For other standards more general to all frameworks we will present the ISO 31000\cite{ISO31000,IEC31010}, a set of standards that provide principles, framework and a process for managing risk, and ISO 19011, a international standard for providing guidelines for auditing management systems. This standards will complement complex areas of project management providing some additional and specific knowledge.

\subsection{ISO/IEC 12207}

As described by ISO/IEC 12207:2008, it ``establishes a common framework for software life cycle processes, with well-defined terminology, that can be referenced by the software industry. It contains processes, activities, and tasks that are to be applied during the acquisition of a software product or service and during the supply, development, operation, maintenance and disposal of software products.''.\cite{ISO12207} The main objective of this standard is to present standardized processes that will make easier the communication among all stakeholders in the software product life cycle.\par
This standard groups the processes to be performed during the life cycle of a software project in seven groups . It presents, for each process, its objectives and expected results as well as the necessary activities to implement it. The processes are grouped in 7 groups, related to the phase they are applied during the software life cycle. All processes are listed in figure 7.

\begin{itemize}
  \item Agreement processes
  \item Organizational Project-Enabling Processes
  \item Project Processes
  \item Technical Processes
  \item Software Implementation Processes
  \item Software Support Processes
  \item Software Reuse Processes 
\end{itemize}

 
\begin{figure}[t!]
\centering
\includegraphics[width=\textwidth]{img/ISO12207Processes.png}
\caption{ISO/IEC 12207 processes}
\end{figure}

It may be used standalone or jointly with ISO/IEC 15288\cite{ISO15288}, an international standard for system life cycle processes, and supplies a process reference model that supports process capability assessment in accordance with ISO/IEC 15504\cite{ISO15504}, a set of technical standards documents for the computer software development processes assessment.\par
This standard is important for this project in the way it standardize the processes for the whole life cycle of the software, grouping the processes for a better understanding its scope. We will use this standard, specifically figure 7, to present the processes that make part of the scope of this project, after what we will relate them with the frameworks previously presented. 


\subsection{ISO/IEC 20000}

ISO/IEC 20000 corresponds to a standard on IT Service Management. Initially was developed to reflect best practice guidance contained in some frameworks like ITIL, COBIT or Microsoft operations. This standard in composed by 5 parts:

\begin{itemize}
  \item \textbf{ISO/IEC 20000-1:2011} -  Corresponds to the most relevant part of the ISO/IEC 20000 standard. It specifies requirements for the service provider to manage the whole system life cycle. Similar to the ITIL V3 view, the requirements include the design, transition, delivery and improvement of services to agree with the service requirements established.\cite{ISO20000-1}\par
  
  \vspace{5mm}
  
  \item \textbf{ISO/IEC 20000-2:2012} - Provides guidance on implementing Service management systems defined by the requirements of ISO/IEC 20000-1. As presented by ISO/IEC 20000-2, ``Enables organizations and individuals to interpret ISO/IEC 20000-1 more accurately, and therefore to use it more effectively. The guidance includes examples and suggestions to enable organizations to interpret and apply ISO/IEC 20000-1, including references to other parts of ISO/IEC 20000 and other relevant standards.''\cite{ISO20000-2}\par
  
  \vspace{5mm}
  
  \item \textbf{ISO/IEC 20000-3:2012} - Used by service providers, consultants and assessors, provides guidance on scope definition, applicability and demonstration of conformity to ISO/IEC 20000-1 requirements specification. It also contains assessment standards.\cite{ISO20000-3}\par
  \vspace{5mm}
  
  \item \textbf{ISO/IEC TR 20000-4:2010} - This standard acts as a facilitator for developing a process assessment model according to ISO/IEC 15504 process assessment principles. Related to ISO/IEC 15504, ISO/IEC 15504-1 describes the concepts and terminology used for process assessment and ISO/IEC 15504-2 describes the requirements for the conduct of an assessment and a measurement scale for assessing process capability.\cite{ISO20000-4}\par
  
  \vspace{5mm}
  
  \item \textbf{ISO/IEC TR 20000-5:2013} - This standard presents an implementation plan on how to implement a service management system to fulfill the requirements of ISO/IEC 20000-1:2011. This standard is planned to be used by service providers but can also be used for his advisors to provide guidance on how to implement an service management system.\cite{ISO20000-5}\par
  
  \vspace{5mm}

  This standard is a clear complement to the ITIL framework, providing a similar view for the framework previous presented but being more complete in terms of requirements identification and process assessment. ITIL lacks of a process assessment model and detailed implementation plans, being this standard a way to fulfill those problems.\par
  
\end{itemize}

\subsection{ISO/IEC 27000}

ISO/IEC 27000 is family of international standards related to Information Security management systems (ISMS). This set of standards intent to help organizations of any size to implement and operate an ISMS. As stated by ISO, this family of standards contain information on:

\begin{itemize}

\item Requirements definition for an ISMS and for certification of those systems.
\item Support and guidance for the overall process to establish, implement, maintain and improve and ISMS.
\item Conformity assessment for ISMS.
\item Terms and definitions related to Information security management.

\end{itemize}

This family of standards is commonly used by organizations to implement frameworks for managing information security, protecting important assets as financial data or customers details. Information security is one of the main concerns on any organization, because information leaks or losses can have severe consequences for the overall organization.\par
International standards as ISO 31000 and ISO 19011 are also related to this family of standards, making risk management and auditing management systems, respectively, areas that have direct impact on information security management systems. Dealing with information security is impossible without considering risk. The overall ISMS need to take account processes for risk identification, treatment and assessment for establishing a secure management system. Furthermore, and also related to risk, ISMS need to consider auditing management as a way to ensure information security and its correct management.\par

This standard will be important for our project considering we deal with processes that consume and produce information, and all information is an important asset for any organization implement this processes, making this standard fundamental to establish an information security management system to protect all the information we deal with.\par



\subsection{ISO 31000}

ISO 31000 is a set of standards related to risk management. ISO 31000:2009, Risk management - Principles and guidelines, provides principles, framework and a process for managing risk. It can be used by any organization independently of its context of operation (size, sector, activity). As presented by this standard, ``Using ISO 31000 can help organizations increase the likelihood of achieving objectives, improve the identification of opportunities and threats and effectively allocate and use resources for risk treatment.''\cite{ISO31000}\par
In figure 8, we can observe the purposed framework by ISO 31000:2009, used to implement risk management on the management system of the organization. Many organizations have already frameworks for implementing risk management, distinct from ISO 31000, but that can be evaluated and reviewed against this international standard in order to check its suitability. In figure 9 are presented the processes for Risk management implementation.\par

\begin{figure}[h!]
\centering
\includegraphics[width=\textwidth]{img/ISO31000Framework.png}
\caption{ISO 31000:2009 framework for Risk Management}
\end{figure}

\begin{figure}[b!]
\centering
\includegraphics[width=0.8\textwidth]{img/ISO31000RiskProcesses.png}
\caption{ISO 31000:2009 Risk Management Processes}
\end{figure}

As confirmed in this standard, it cannot be used for certification purposes, in the way it doesn't provide guidance for audit programmes, but can be used to compare the risk management processes of the organization with international standard benchmarks.\par
Risk management is a very complex area on project management. It is connected to the whole life cycle of a project and need to be controlled and assessed in many phases and by different forms. IEC 31010:2009, Risk management - Risk assessment techniques focuses on risk assessment. Risk assessment helps organizations understand the risks that could jeopardize the achievement of management and governance goals as well as the suitability of the risk control activities already in place.\cite{IEC31010}\par
The relevance of ISO 31000 for this project focuses on its orientation to an important and complex area of project and maintenance management, providing a framework for Risk management, as well as a set of processes to implement it. It is important to clearly define the scope of this standard related to this project, trying to reduce the complexity and increase its utility for the processes we pretend to design.


\subsection{ISO 19011}

As defined by this International Standard, ``ISO 19011 provides guidance on auditing management systems, including the principles of auditing, managing an audit programme and conducting management system audits, as well as guidance on the evaluation of competence of individuals involved in the audit process, including the person managing the audit programme, auditors and audit teams. It is applicable to all organizations that need to conduct internal or external audits of management systems or manage an audit programme.''\cite{ISO19011}\par
In figure 10 is presented the process flow for the management of an audit programme, described in this international standard. This process can be specially important for our project if we pretend to introduce the auditing processes in the scope of the processes to design. We can use the process flow presented in figure 10 to help designing a auditing process.

\begin{figure}[t!]
\centering
\includegraphics[width=0.85\textwidth]{img/ISO19011AuditProcess.png}
\caption{ISO 19011 process flow for the management of an audit programme}
\end{figure}


\section{A State of the Art in Market Solutions}

Considering we are interested in developing a logical application architecture, we need to analyze solutions available on the market to compose this architecture. This solutions should be evaluated accordingly with its importance and benefit for project purposes, namely for improving our performance in project management and evolving maintenance.\par
Considering this two areas of interest for this project, we will perform an analysis of the solutions available on the market for Project and Portfolio Management Tools (PPM Tools) and for IT Service Support Tools (ITSM Tools). There are hundreds of solutions available for both tools, so we need find mechanisms to evaluate some of them fairly and taking in account its functionalities and benefits for this project. Thus, we will use researches performed in this areas by Gartner and and Forrester, teo companies dedicated on IT research and advisory. Both companies provide techniques to evaluate PPM and ITSM tools, providing us the necessary information to evaluate the market offers and perform a usage proposal based on this results.\par

\subsection{Gartner Research}

Gartner inc. is an information technology and research and advisory company that delivers to its clients technology-related insight for helping on business decisions. As stated by Gartner, ``From CIOs and senior IT leaders in corporations and government agencies, to business leaders in high-tech and telecom enterprises and professional services firms, to technology investors, we are the valuable partner to clients in over 9,100 distinct enterprises worldwide.''\par
Gartner provides a set of methodologies to help clients on its IT investments, reducing and managing risks as well as helping achieving sucess through research processes that are based in years of experience and maturation. The objective of gartner is to provide methodologies that ensure business decisions by clients are made with high levels of confidence.\par
Considering this project objectives, we will use the following methodologies from gartner:

\subsubsection{Gartner Magic Quadrant}

The objective of the Gartner Magic Quadrant is to provide a wide-angle view of the relative positions of a specific market's competitors. It helps on analyzing how well technology providers are executing against their stated vision, presenting a graphical competitive positioning (see figure XXX) of four types of technology providers, in growing markets with providers differentiation. The four types of technology providers are:

\begin{itemize}
\item Leaders - Execute well against their current vision and are well positioned for tomorrow.
\item Visionaries - understand where the market is going or have a vision for changing market rules, but do not yet execute well.
\item Niche Players - Focus successfully on a small segment, or are unfocused and do not out-innovate or outperform others.
\item Challengers - Execute well today or may dominate a large segment, but do not demonstrate an understanding of market direction.
\end{itemize}

\begin{figure}[t!]
\centering
\includegraphics[width=0.85\textwidth]{img/GartnerMagicQuadrant.png}
\caption{Gartner Magic Quadrant representation}
\end{figure}

This methodology is fundamental to analyze technology providers we can consider for adoption, but we need to take in account on how each provider aligns with the business goals of the organization. Not always focusing on the leaders' quadrant is the best option and we need to know what specific objectives we want to achieve to select the best provider to fulfill our requirements. 


\subsubsection{Gartner MarketScope}

Gartner MarketScope methodology is used for analyzing solutions for emerging markets with changing requirements. Gartner Magic Quadrant is not suitable for this type of markets as far as a competitive positioning is not useful and accurate for an emerging market. Thus, MarketScope intents to understand how the status of an emerging market aligns with clients state of maturity and future plans. It helps evaluating participating technology providers against the Gartner vision for that market. MarketScopes rate each market’s participating technology providers as Strong Positive, Positive, Promising, Caution or Strong Negative.\par
This methodology is complementing the Magic Quadrant methodology being more suitable to emerging markets, where clients are less interested on technology providers positioning on market but on the investment opportunity on an emerging market solution and the risks associated with it.\par 

\subsubsection{Gartner Critical Capabilities}

This methodology will complement Gartner Magic Quadrant, providing deeper insight into providers' solutions, presenting its offerings based on service and product ratings of key capabilities considered as the most important for a specific service or product. As stated by Gartner, ``a Critical Capabilities document is a comparative analysis that scores competing products or services against a set of critical differentiators identified by Gartner. It shows you which products or services are a best fit in various use cases to provide you actionable advice on which products/services you should add to your vendor shortlists for further evaluation.''\par
This methodology offers better benefits when applied in line with the Magic Quadrant methodology, providing deeper insight on products and services offerings from providers that Magic Quadrant positions on the market. When available, we will consider these two methodologies together for evaluating PPM and ITSM solutions.\par

\subsection{Forrester Research} 

Forrester is an research and advisory company providing guidance to professionals in 13 key roles across the business technology, marketing and strategy segments. It presents to its clients research-based services to help them on business needs and decisions, centering its corporate mission on providing client's adaptable and centered solutions to his needs. As stated by Forrester Role Manifesto, its core values, called 3CIQ, are the Clients, Collaboration, Courage, Integrity and Quality.\par
Considering our objectives for this project, we will consider the Forrester Wave Methodology to analyze the PPM and ITSM solutions available on the market.

\subsubsection{Forrester Wave Methodology}

This methodology is provided by Forrester to evaluate vendors in software, hardware or service market, presenting the criteria and the weights for this evaluation. this evaluation methodology is transparent for the client, being only based in criteria weighting performed by Forrester research professionals. In addition, all the methodology process and participants are presented, making clear how the evaluation was conducted.\par
Each Forrester Wave counts with the participation of four key elements: An analyst, a research associate, a vendor response team and costumer references. Before starting the analysis, the analyst performs a preparation process, researching the evaluation category, identifying the category and the scope, selecting and evaluation method and creating the research plan and timeline.\par
Consider the evaluation process, it is composed by five milestones: Create evaluation criteria, determine vendors for inclusion, gather evaluation data, create the vendor comparison and publish the report. This milestones ensure a well defined and independent process.\par
We will present, for the PPM an ITSM tools, the corresponding Forrester Wave reports, complementing the Gartner evaluation with Forrester analysis making our research for this tools more valuable and rich.\par 

\subsection{Project and Portfolio Management Tools analysis}

\subsubsection{Gartner Magic Quadrant for PPM solutions}

\subsubsection{Gartner Magic Quadrant for Cloud-Based Solutions}

\subsubsection{Gartner MarketScope}

\subsubsection{Forrester Wave}

\subsection{IT Service Support Management Tools analysis}

\subsubsection{Gartner Magic Quadrant for ITSM solutions}

\subsubsection{Gartner Critical Capabilities Cloud-Based Solutions}

\subsubsection{Forrester Wave}




